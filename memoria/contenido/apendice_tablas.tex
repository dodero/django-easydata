En este apartado se van a plasmar cada una de las tablas que se han desarrollado
para el apartado de análisis de requisitos del proyecto.

\section{Tablas de requisitos funcionales}

A continuación se muestran cada una de las tablas donde se indican cada uno de
los requisitos de funcionales que debe de implementar el proyecto.

\subsection{Carga de modelos}

%%%TABLA - Caso de uso 001
\begin{center}
\begin{longtable}{||p{3.4cm}|p{12cm}||}
%primera parte de la tabla
 \hline \hline \bf UC-001 &  \bf Carga de modelos \\
\hline
\endfirsthead
%primera parte de la tabla por pagina
\hline \multicolumn{2}{|r|}{{Continuación de la tabla}} \\ \hline
 \hline \bf UC-001 &  \bf Carga de modelos \\
\hline
\endhead
% ultima parte de la tabla por pagina
\hline \multicolumn{2}{|l|}{{Continúa en la siguiente página}} \\ \hline
\endfoot
% ultima parte de la tabla
\endlastfoot
% DATOS
 \hline \bf Autor & José Manuel Llerena Carmona \\
 \hline \bf Descripción & Carga de modelos y fields del proyecto Django
             donde se va a usar la aplicación.\\
 \hline \bf Actores & Administrador del sistema\\
 \hline \bf Precondición & Se ha configurado la aplicación en el proyecto
             Django, la aplicación posee una base de datos y se han sincronizado
             las modelos de la aplicación con la base de datos de la aplicación
             para que se creen las tablas necesarias.\\
 \hline \bf Postcondición & Se han cargado en las tablas de la base de datos
             correspondiente a la aplicación los modelos y fields de estos.\\
 \hline \bf Secuencia normal & 
             \begin{enumerate}
                \item El administrador accede a través de la línea de comandos
                       de la terminal al path donde se encuentra el proyecto Django.
                \item El administrador ejecuta el script que se encarga de
                       cargar los modelos y fields existentes en la aplicación
                       Django.
                \item El sistema indica al administrador que los datos se han
                       cargado correctamente.
             \end{enumerate}\\
 \hline \bf Excepciones &
             \begin{description}
                \item[3.1] Existen modelos ya cargados en la base de datos, por
                            lo que el sistema actualiza dichas tablas y añade
                            las que no existan.
                \item[3.2] La aplicación no tiene configurada una base de datos
                            o sus modelos no están configurados correctamente.
                            La aplicación aborta el procedimiento de carga de
                            modelos y muestra mensaje de error.
             \end{description}\\
\hline
\hline
\caption{\label{tab:caso001} Caso de Uso - 001 - Carga de modelos} 
\end{longtable}
\end{center}


\subsection{Carga de espacios de nombres}

%%%TABLA - Caso de uso 002
\begin{center}
\begin{longtable}{||p{3.4cm}|p{12cm}||}
%primera parte de la tabla
 \hline \hline \bf UC-002 &  \bf Carga de espacios de nombres \\
\hline
\endfirsthead
%primera parte de la tabla por pagina
\hline \multicolumn{2}{|r|}{{Continuación de la tabla}} \\ \hline
 \hline \bf UC-002 &  \bf Carga de espacio de nombres \\
\hline
\endhead
% ultima parte de la tabla por pagina
\hline \multicolumn{2}{|l|}{{Continúa en la siguiente página}} \\ \hline
\endfoot
% ultima parte de la tabla
\endlastfoot
% DATOS
 \hline \bf Autor & José Manuel Llerena Carmona \\
 \hline \bf Descripción & Se debe de realizar una serie de procedimientos los
             cuales permitan a la aplicación cargar la estructura de los
             diferentes namespaces existentes, además de permitir actualizar y
             eliminar estos.\\
 \hline \bf Actores & Administrador del sistema\\
 \hline \bf Precondición & Se ha configurado la aplicación en el proyecto
             Django, la aplicación posee una base de datos y se han sincronizado
             las modelos de la aplicación con la base de datos de la aplicación
             para que se creen las tablas necesarias.\\
 \hline \bf Postcondición & Se han cargado en las tablas de la base de datos
             correspondientes a la aplicación las distintas entidades que
             componen al namespaces y los atributos de cada una de estas
             entidades.\\
 \hline \bf Secuencia normal & 
             \begin{enumerate}
                \item El administrador sistema accede al listado de namespaces
                       disponibles.
                \item El sistema muestra el listado de namespaces existentes en
                       la aplicación.
                \item El administrador del sistema selecciona crear un nuevo
                       namespace.
                \item El sistema muestra al administrador un formulario para
                       crear un nuevo namespace.
                \item El administrador del sistema indica los datos del nuevo
                       namespace, así como el fichero o la URL donde se
                       encuentra la especificación de dicho namespace.
                \item El sistema carga la especificación del namespace en la
                       base de datos, y muestra por pantalla el resultado de la
                       carga.
             \end{enumerate}\\
 \hline \bf Excepciones &
             \begin{description}
                \item[*.1] El administrador del sistema puede cancelar en
                          cualquier momento el procedimiento de
                          carga/actualización de los namespaces.
                \item[*.2] El usuario no tiene permisos de superusuario.
                \item[3.1] El usuario indica editar un namespace, para
                          actualizar su nombre o actualizar la especificación
                          mediante un fichero o URL.
                \item[6.1] La URL o fichero especificado por el usuario no es
                          correcta, y no se encuentra en ella una estructura de
                          namespaces, o dichos datos no son correctos.
                \item[6.2] Los nombres suministrados para el namespace ya están
                          siendo usados. El sistema vuelve a solicitar un nombre
                          correcto.
             \end{description}\\
\hline
\hline
\caption{\label{tab:caso002} Caso de Uso - 002 - Carga de espacios de nombres}
\end{longtable}
\end{center}


\subsection{Mapeo de modelos}

%%%TABLA - Caso de uso 003
\begin{center}
\begin{longtable}{||p{3.4cm}|p{12cm}||}
%primera parte de la tabla
 \hline \hline \bf UC-003 &  \bf Mapeo de modelos \\
\hline
\endfirsthead
%primera parte de la tabla por pagina
\hline \multicolumn{2}{|r|}{{Continuación de la tabla}} \\ \hline
 \hline \bf UC-003 &  \bf Mapeo de modelos \\
\hline
\endhead
% ultima parte de la tabla por pagina
\hline \multicolumn{2}{|l|}{{Continúa en la siguiente página}} \\ \hline
\endfoot
% ultima parte de la tabla
\endlastfoot
% DATOS
 \hline \bf Autor & José Manuel Llerena Carmona \\
 \hline \bf Descripción & Este caso de uso describe el procedimiento mediante
             el cual el administrador del sistema, describe en la aplicación la
             relación de cada uno de los modelos del proyecto Django, con las
             entidades de los distintos namespaces que existen en la
             aplicación.\\
 \hline \bf Actores & Administrador del sistema\\
 \hline \bf Precondición & Se ha cargado en la aplicación los modelos y
             atributos existentes en el proyecto, y además existe algún namespace
             cargado en la aplicación, de forma que se dispongan entidades y
             propiedades con el que mapear los modelos.\\
 \hline \bf Postcondición & Uno o varios de los modelos del proyecto Django de
             los cuales se quieren publicar sus datos, se encuentran
             relacionados con una de las entidades de los namespaces
             existentes.\\
 \hline \bf Secuencia normal & 
             \begin{enumerate}
                \item El administrador del sistema accede al listado de
                       modelos del sistema disponibles para mapear.
                \item El sistema muestra el listado de modelos y dos selectores
                       por cada uno de los modelos, donde se pueda elegir el
                       namespace con el que se desea mapea y la entidad concreta
                       con la que se desea realizar el mapeo.
                \item El administrador del sistema introduce la configuración de
                       aquello modelos que crea oportunos y salva los cambios
                       realizados.
                \item El sistema muestra un mensaje con el resultado del mapeo.
             \end{enumerate}\\
 \hline \bf Excepciones &
             \begin{description}
                \item[*.1] El administrador del sistema puede cancelar en
                          cualquier momento la operación de mapeo de los modelos.
                \item[*.2] El usuario no tiene permisos de superusuario.
             \end{description}\\
\hline
\hline
\caption{\label{tab:caso003} Caso de Uso - 003 - Mapeo de modelos} 
\end{longtable}
\end{center}


\subsection{Mapeo de los atributos}

%%%TABLA - Caso de uso 004
\begin{center}
\begin{longtable}{||p{3.4cm}|p{12cm}||}
%primera parte de la tabla
 \hline \hline \bf UC-004 &  \bf Mapeo de los atributos \\
\hline
\endfirsthead
%primera parte de la tabla por pagina
\hline \multicolumn{2}{|r|}{{Continuación de la tabla}} \\ \hline
 \hline \bf UC-004 &  \bf Mapeo de los atributos \\
\hline
\endhead
% ultima parte de la tabla por pagina
\hline \multicolumn{2}{|l|}{{Continúa en la siguiente página}} \\ \hline
\endfoot
% ultima parte de la tabla
\endlastfoot
% DATOS
 \hline \bf Autor & José Manuel Llerena Carmona \\
 \hline \bf Descripción & Este caso de uso describe el procedimiento mediante
             el cual el administrador del sistema, describe en la aplicación la
             relación de cada uno de los atributos de los modelos del proyecto
             Django, con las propiedades de las entidades de los distintos
             namespaces que existen en la aplicación.\\
 \hline \bf Actores & Administrador del sistema\\
 \hline \bf Precondición & Se ha cargado en la aplicación los modelos y
             atributos existentes en el proyecto, además existe algún namespace
             cargado en la aplicación, con el que mapear la relación con los
             atributos y además, se ha realizado la correspondencia del modelo
             con alguna entidad del namespace que deseamos configurar.\\
 \hline \bf Postcondición & Alguno de los atributos del modelo del proyecto
             Django que se ha configurado, ha sido relacionado con alguna de las
             propiedades de la entidad con la que está relacionada el modelo.\\
 \hline \bf Secuencia normal & 
             \begin{enumerate}
                \item El administrador del sistema accede al listado de
                       modelos del sistema disponibles para mapear.
                \item El sistema muestra el listado de modelos.
                \item El administrador del sistema elige configurar los
                       atributos de un determinado modelo que ya se encuentra
                       mapeado con alguna entidad.
                \item El sistema una pantalla con todos los atributos del modelo
                       y dos selectores con los posibles namespaces y las
                       propiedades disponibles del namespace seleccionado
                       disponibles para mapear dicho atributo.
                \item El administrador del sistema, introduce la configuración
                       de aquellos atributos que desee y salva los cambios.
                \item El sistema muestra un mensaje con el resultado del mapeo.
             \end{enumerate}\\
 \hline \bf Excepciones &
             \begin{description}
                \item[*.1] El administrador del sistema puede cancelar en
                          cualquier momento la operación de mapeo de los
                          atributos.
                \item[*.2] El usuario no tiene permisos de superusuario.
                \item[3] El modelo no se encuentra mapeado en el sistema, por lo
                          que previamente deberá de realizar el mapeo del mismo
                          antes de configurar el mapeo de sus atributos.
                          \textbf{Include:} UC-003.
             \end{description}\\
\hline
\hline
\caption{\label{tab:caso004} Caso de Uso - 004 - Mapeo de los atributos} 
\end{longtable}
\end{center}


\subsection{Establecer visibilidad de modelos}

%%%TABLA - Caso de uso 005
\begin{center}
\begin{longtable}{||p{3.4cm}|p{12cm}||}
%primera parte de la tabla
 \hline \hline \bf UC-005 &  \bf Establecer visibilidad de modelos \\
\hline
\endfirsthead
%primera parte de la tabla por pagina
\hline \multicolumn{2}{|r|}{{Continuación de la tabla}} \\ \hline
 \hline \bf UC-005 &  \bf Establecer visibilidad de modelos \\
\hline
\endhead
% ultima parte de la tabla por pagina
\hline \multicolumn{2}{|l|}{{Continúa en la siguiente página}} \\ \hline
\endfoot
% ultima parte de la tabla
\endlastfoot
% DATOS
 \hline \bf Autor & José Manuel Llerena Carmona \\
 \hline \bf Descripción & Este caso de uso describe el procedimiento que debe
             de seguir el administrador del sistema, para especificar aquellos
             modelos del proyecto Django de los que podrán publicarse los datos.\\
 \hline \bf Actores & Administrador del sistema\\
 \hline \bf Precondición & Se ha configurado la aplicación en el proyecto
             Django, la aplicación posee una base de datos y se han sincronizado
             las modelos de la aplicación con la base de datos de la aplicación
             para que se creen las tablas necesarias donde se almacenará la
             estructura de los namespaces.\\
 \hline \bf Postcondición & Se han marcado aquellos modelos que queremos que se
             puedan mostrar sus datos y los que no queremos que se puedan
             mostrar como visibles y no visibles, respectivamente.\\
 \hline \bf Secuencia normal & 
             \begin{enumerate}
                \item El administrador del sistema accede al apartado de
                       configuración de visibilidad de modelos.
                \item El sistema muestra una lista con todos los modelos y
                       fields de los mismos existentes en el proyecto Django.
                \item El administrador del sistema marca como visibles o no
                       visibles, los modelos y fields que desea que se puedan
                       ver o no.
                \item El administrador del sistema una vez ha terminado, salva
                       los cambios en el sistema.
                \item El sistema indica al administrador que los datos se han
                       cargado correctamente.
             \end{enumerate}\\
 \hline \bf Excepciones &
             \begin{description}
                \item[*.1] El administrador del sistema puede cancelar en
                          cualquier momento la operación de configuración de
                          visibilidad de los modelos.
                \item[*.2] El usuario no tiene permisos de superusuario.
             \end{description}\\
\hline
\hline
\caption{\label{tab:caso005} Caso de Uso - 005 - Establecer visibilidad de modelos} 
\end{longtable}
\end{center}


\subsection{Establecer visibilidad de los atributos}

%%%TABLA - Caso de uso 006
\begin{center}
\begin{longtable}{||p{3.4cm}|p{12cm}||}
%primera parte de la tabla
 \hline \hline \bf UC-006 &  \bf Establecer visibilidad de los atributos\\
\hline
\endfirsthead
%primera parte de la tabla por pagina
\hline \multicolumn{2}{|r|}{{Continuación de la tabla}} \\ \hline
 \hline \bf UC-006 &  \bf Establecer visibilidad de los atributos \\
\hline
\endhead
% ultima parte de la tabla por pagina
\hline \multicolumn{2}{|l|}{{Continúa en la siguiente página}} \\ \hline
\endfoot
% ultima parte de la tabla
\endlastfoot
% DATOS
 \hline \bf Autor & José Manuel Llerena Carmona \\
 \hline \bf Descripción & Este caso de uso describe el procedimiento que debe
             de seguir el administrador del sistema, para especificar aquellos
             atributos de los modelos del proyecto Django de los que podrán
             publicarse los datos.\\
 \hline \bf Actores & Administrador del sistema\\
 \hline \bf Precondición & Se ha configurado la aplicación en el proyecto
             Django, la aplicación posee una base de datos y se han sincronizado
             las modelos de la aplicación con la base de datos de la aplicación
             para que se creen las tablas necesarias donde se almacenará la
             estructura de los namespaces.\\
 \hline \bf Postcondición & Se han marcado aquellos atributos de los modelos
             que queremos que se puedan mostrar sus datos y los que no queremos
             que se puedan mostrar como visibles y no visibles, respectivamente.\\
 \hline \bf Secuencia normal & 
             \begin{enumerate}
                \item El administrador del sistema accede al apartado de
                       configuración de visibilidad de modelos.
                \item El sistema muestra una lista con todos los modelos
                       existentes en el proyecto Django.
                \item El administrador del sistema selecciona en configuración
                       de los atributos de un determinado modelo.
                \item El sistema muestra al usuario la lista de atributos del
                       modelo seleccionado.
                \item El administrador del sistema marca como visibles o no
                       visibles, los atributos de los modelos que desea que se
                       puedan ver o no, respectivamente.
                \item El administrador del sistema una vez ha terminado, salva
                       los cambios en el sistema.
                \item El sistema indica al administrador que los datos se han
                       cargado correctamente.
             \end{enumerate}\\
 \hline \bf Excepciones &
             \begin{description}
                \item[*.1] El administrador del sistema puede cancelar en
                          cualquier momento la operación de configuración de
                          visibilidad de los modelos.
                \item[*.2] El usuario no tiene permisos de superusuario.
             \end{description}\\
\hline
\hline
\caption{\label{tab:caso006} Caso de Uso - 006 - Establecer visibilidad \mbox{atributos}} 
\end{longtable}
\end{center}


\subsection{Publicación de los datos}

%%%TABLA - Caso de uso 007
\begin{center}
\begin{longtable}{||p{3.4cm}|p{12cm}||}
%primera parte de la tabla
 \hline \hline \bf UC-007 &  \bf Consulta de datos \\
\hline
\endfirsthead
%primera parte de la tabla por pagina
\hline \multicolumn{2}{|r|}{{Continuación de la tabla}} \\ \hline
 \hline \bf UC-007 &  \bf Consulta de datos \\
\hline
\endhead
% ultima parte de la tabla por pagina
\hline \multicolumn{2}{|l|}{{Continúa en la siguiente página}} \\ \hline
\endfoot
% ultima parte de la tabla
\endlastfoot
% DATOS
 \hline \bf Autor & José Manuel Llerena Carmona \\
 \hline \bf Descripción & Muestra los datos por pantalla usando alguno de los
             estándares existente para la publicación de datos en internet.\\
 \hline \bf Actores & Usuario externo\\
 \hline \bf Precondición & Se ha configurado la aplicación en el proyecto
             Django, la aplicación posee una base de datos y se han sincronizado
             las modelos de la aplicación con la base de datos de la aplicación
             para que se creen las tablas necesarias. Además se ha realizado el
             mapeo el modelo junto con sus atributos que se desea visualizar y
             también se ha configurado su visibilidad.\\
 \hline \bf Postcondición & Se ha mostrado por pantalla usando el formato
             deseado los datos referente a la instancia(s) del modelo
             seleccionado.\\
 \hline \bf Secuencia normal & 
             \begin{enumerate}
                \item El usuario externo accede a la URI correspondiente a la
                       dirección dinámica existente para los datos que desea
                       consultar.
                \item El sistema muestra al usuario externo los datos del
                       modelo o los modelos , haciendo uso del estándar y el
                       namespace solicitado.
             \end{enumerate}\\
 \hline \bf Excepciones &
             \begin{description}
                \item[*] El usuario externo puede cancelar la operación en
                          cualquier momento.
                \item[2.1] El sistema muestra los datos de la instancia concreta
                          en el formato solicitado.
                \item[2.2] El sistema muestra los datos de la entidad concreta en
                          el formato solicitado.
             \end{description}\\
\hline
\hline
\caption{\label{tab:caso007} Caso de Uso - 007 - Consulta de datos} 
\end{longtable}
\end{center}


\subsection{Inicio de sesión}

%%%TABLA - Caso de uso 008
\begin{center}
\begin{longtable}{||p{3.4cm}|p{12cm}||}
%primera parte de la tabla
 \hline \hline \bf UC-008 &  \bf Inicio de sesión \\
\hline
\endfirsthead
%primera parte de la tabla por pagina
\hline \multicolumn{2}{|r|}{{Continuación de la tabla}} \\ \hline
 \hline \bf UC-008 &  \bf Inicio de sesión \\
\hline
\endhead
% ultima parte de la tabla por pagina
\hline \multicolumn{2}{|l|}{{Continúa en la siguiente página}} \\ \hline
\endfoot
% ultima parte de la tabla
\endlastfoot
% DATOS
 \hline \bf Autor & José Manuel Llerena Carmona \\
 \hline \bf Descripción & Realiza el login del usuario dentro de la aplicación
        web.\\
 \hline \bf Actores & Usuario externo\\
 \hline \bf Precondición & El usuario externo no se encuentra logueado en la
        aplicación.\\
 \hline \bf Postcondición & Se ha iniciado sesión en la aplicación y tiene
            permisos de superusuario.\\
 \hline \bf Secuencia normal & 
             \begin{enumerate}
                \item El usuario externo accede a la dirección de inicio de
                      sesión.
                \item El sistema muestra al usuario externo un formulario donde
                      se le solicita el usuario y contraseña.
                \item El usuario introduce las credenciales.
                \item El usuario y contraseña son válidos y el sistema redirige
                      al usuario a la aplicación.
             \end{enumerate}\\
 \hline \bf Excepciones &
             \begin{description}
                \item[*] El usuario externo puede cancelar la operación en
                          cualquier momento.
                \item[4.1] El usuario o contraseña son incorrectos. Vuelve a
                      solicitar las credenciales al usuario.
                \item[4.2] El usuario no posee credenciales de superusuario.
             \end{description}\\
\hline
\hline
\caption{\label{tab:caso008} Caso de Uso - 008 - Inicio de sesión} 
\end{longtable}
\end{center}


\subsection{Consulta con SPARQL}

%%%TABLA - Caso de uso 009
\begin{center}
\begin{longtable}{||p{3.4cm}|p{12cm}||}
%primera parte de la tabla
 \hline \hline \bf UC-009 &  \bf Consulta con SPARQL \\
\hline
\endfirsthead
%primera parte de la tabla por pagina
\hline \multicolumn{2}{|r|}{{Continuación de la tabla}} \\ \hline
 \hline \bf UC-009 &  \bf Consulta con SPARQL \\
\hline
\endhead
% ultima parte de la tabla por pagina
\hline \multicolumn{2}{|l|}{{Continúa en la siguiente página}} \\ \hline
\endfoot
% ultima parte de la tabla
\endlastfoot
% DATOS
 \hline \bf Autor & José Manuel Llerena Carmona \\
 \hline \bf Descripción & Debe de permitirse al usuario realizar consultas sobre
        los datos utilizando el lenguaje SPARQL.\\
 \hline \bf Actores & Usuario externo\\
 \hline \bf Precondición & Los modelos y fields se han cargado en la aplicación.
            Se han cargado también algún namespace. Se ha realizado el mapeo de
            alguno de los modelos del proyecto.\\
 \hline \bf Postcondición & Se devuelven aquellos datos que se corresponde con
            la consulta SPARQL indicada.\\
 \hline \bf Secuencia normal & 
             \begin{enumerate}
                \item Un usuario externo accede al apartado de consultas SPARQL
                      e introduce una consulta.
                \item El sistema devuelve al usuario el resultado de ejecutar la
                      consulta sobre los datos almacenados en la base de datos y
                      que cumplan las relaciones indicadas en la consulta.
             \end{enumerate}\\
 \hline \bf Excepciones & No existen excepciones para este caso de uso.\\
\hline
\hline
\caption{\label{tab:caso009} Caso de Uso - 009 - Generación fichero D2Rq} 
\end{longtable}
\end{center}


\section{Tablas de requisitos de información}

A continuación se muestran cada una de las tablas donde se indican cada uno de
los requisitos de información que posee el proyecto.

%%%TABLA - Requisito Información de los namespaces
\begin{center}
\begin{longtable}{||p{3.4cm}|p{12cm}||}
%primera parte de la tabla
 \hline \hline \bf IRQ-001 &  \bf Información de los namespaces \\
\hline
\endfirsthead
%primera parte de la tabla por pagina
\hline \multicolumn{2}{|r|}{{Continuación de la tabla}} \\ \hline
 \hline \bf IRQ-001 &  \bf Información de los namespaces \\
\hline
\endhead
% ultima parte de la tabla por pagina
\hline \multicolumn{2}{|l|}{{Continúa en la siguiente página}} \\ \hline
\endfoot
% ultima parte de la tabla
\endlastfoot
% DATOS
 \hline \bf Autor & José Manuel Llerena Carmona \\
 \hline \bf Descripción & Se debe almacenar la información acerca de los
             distintos namespaces que almacena la aplicación, de tal forma que
             se tenga la información acerca del namespace, de las entidades que
             componen a este, y de las propiedades que componen a la entidad.\\
 \hline \bf Dependencias & De este requisito de información depende que se
                            pueda almacenar la información relativa a los
                            distintos namespaces, realizar el posterior mapeo de
                            los datos y finalmente la publicación de los mismos,
                            usando dichos namespaces.\\
 \hline \bf Datos específicos &
             \begin{itemize} 
                 \item Datos de la clase \textit{Namespace}:
                 \begin{itemize}
                    \item Nombre: nombre identificativo del namespace.
                    \item url: es la dirección donde se encuentra la
                                especificación del namespace.
                    \item short\_name: nombre corto que se representará al
                                       namespace en los ficheros rdf.
                 \end{itemize}
                 \item Datos de la clase \textit{Entidad}:
                 \begin{itemize}
                    \item nombre: es el nombre identificativo de la entidad.
                    \item padres: indica si la entidad hereda de una entidad
                                  padre, y en caso de ser cierto, indica cuáles son.
                    \item namespace: indica el namespace al que pertenece dicha
                                      entidad.
                    \item descripción: descripción asociada a la entidad.
                    \item etiqueta: etiqueta asociada a la entidad, la cual
                        representa al elemento.
                 \end{itemize}
                 \item Datos de la clase \textit{Propiedad}:
                 \begin{itemize}
                    \item nombre: es el nombre identificativo de la propiedad.
                    \item tipo: en caso de que el tipo de dato no sea simple,
                          indica las entidades a las que hace referencia la
                          propiedad.
                    \item entidades: son las entidades a las que pertenece dicha
                          propiedad.
                    \item descripción: descripción asociada a la propiedad.
                    \item etiqueta: etiqueta asociada a la propiedad, la cual
                          representa a dicha característica.
                    \item simple: indica si la propiedad es un tipo de dato
                          simple o si por el contrario hace referencia a otra
                          entidad.
                    \item namespace: indica el namespace al que pertenece la
                          propiedad.
                 \end{itemize}
             \end{itemize}\\
\hline
\hline
\caption{\label{tab:irq001} IRQ - 001 - Información de los namespaces} 
\end{longtable}
\end{center}

%%%TABLA - Requisito Información de los modelos
\begin{center}
\begin{longtable}{||p{3.4cm}|p{12cm}||}
%primera parte de la tabla
 \hline \hline \bf IRQ-002 &  \bf Información de los modelos \\
\hline
\endfirsthead
%primera parte de la tabla por pagina
\hline \multicolumn{2}{|r|}{{Continuación de la tabla}} \\ \hline
 \hline \bf IRQ-002 &  \bf Información de los modelos \\
\hline
\endhead
% ultima parte de la tabla por pagina
\hline \multicolumn{2}{|l|}{{Continúa en la siguiente página}} \\ \hline
\endfoot
% ultima parte de la tabla
\endlastfoot
% DATOS
 \hline \bf Autor & José Manuel Llerena Carmona \\
 \hline \bf Descripción & Se debe almacenar la información acerca de los
             distintos modelos que componen al proyecto Django, de tal forma que
             se tenga la información acerca del modelo, y de los fields que
             componen al modelo.\\
 \hline \bf Dependencias & De este requisito de información, depende que se
                            puedan almacenar la estructura de los modelos del
                            proyecto Django donde se ha instalado la aplicación,
                            su posterior mapeo con los distintos namespaces y su
                            publicación final.\\
 \hline \bf Datos específicos &
             \begin{itemize} 
                 \item Datos de la clase \textit{Modelo}:
                 \begin{itemize}
                    \item Nombre: nombre identificativo del Modelo dentro del
                                   proyecto Django.
                    \item aplicacion: es la aplicación dentro del proyecto
                                       Django al que pertenece dicho modelo. Se
                                       utiliza para poder localizarlo más
                                       adelante, cuando se necesite captar sus
                                       datos.
                    \item Entidad: se trata de una relación muchos-muchos con
                                    cada una de las entidades con las que se ha
                                    configurado su mapeo.
                    \item visibilidad: indica si este modelo es visible o no al
                                        exterior.
                 \end{itemize}
                 \item Datos de la clase \textit{Field}:
                 \begin{itemize}
                    \item nombre: es el nombre identificativo del field dentro
                                   del modelo.
                    \item visibilidad: indica si este field es visible o no al
                                        exterior.
                    \item propiedad: es una relación muchos-muchos donde la
                                      cual almacena el mapeo de dicho field con
                                      las distintas propiedades.
                    \item modelo: indica el modelo del proyecto Django al que
                                   pertenece dicho field.
                 \end{itemize}
                 Esta estará diferenciada a su ver, mediante herencia en dos
                 clases diferentes:
                 \begin{itemize}
                     \item \textbf{Atributo:} se encargará de representar a las
                        relaciones existentes entre los distintos modelos. Posee
                        los siguientes atributos:
                     \begin{itemize}
                         \item tipo de field: almacena el nombre del tipo de
                         atributo al que representa.
                     \end{itemize}
                     \item \textbf{Relacion:} se encargará de representar a las
                        relaciones existentes entre los distintos modelos. Posee
                        los siguientes atributos:
                     \begin{itemize}
                         \item Tipo de relación: indica si la relación es 1:1,
                            1:N o N:M.
                         \item Modelo relacionado: indica el modelo al que hace
                            referencia la relación.
                         \item inversa: indica la relación inversa, es decir, la
                            que posee la navegabilidad en el sentido contrario.
                     \end{itemize}
                 \end{itemize}
             \end{itemize}\\
\hline
\hline
\caption{\label{tab:irq002} IRQ - 002 - Información de los modelos} 
\end{longtable}
\end{center}
