% ------------------------------------------------------------------------------
% Este fichero es parte de la plantilla LaTeX para la realización de Proyectos
% Final de Grado, protegido bajo los términos de la licencia GFDL.
% Para más información, la licencia completa viene incluida en el
% fichero fdl-1.3.tex

% Copyright (C) 2012 SPI-FM. Universidad de Cádiz
% ------------------------------------------------------------------------------

A lo largo del siguiente documento, se describirán las motivaciones que han
llevado a la realización del proyecto, así como los objetivos, estructura,
requisitos y demás características técnicas, implementación y pruebas del mismo,
además de los manuales necesarios para su utilización.

Como punto inicial, en el presente capitulo se describen la motivaciones para la
realización del proyecto, los objetivos y cómo se encuentra estructurada la
documentación.

\section{Motivación}

Actualmente existe una gran cantidad de datos distribuidos por toda la red, pero
éstos no se encuentran publicados ni organizados de forma que otros usuarios u
organizaciones puedan hacer uso de ellos, o mejor dicho, de forma que otras
aplicaciones externas puedan entenderlos. Especificando un formato común para
publicar dicha información, se podrían obtener grandes cantidades de información
en la web, a disposición de cualquier usuario u organización que la desease.

Hoy en día muchos Gobiernos (OpenGovernment) están abriendo parte de sus datos
en la Web, beneficiando así a multitud de usuarios, los cuales pueden captar
estos datos y generar estadísticas, informes, estudios, generando transparencia
en la gestión del gobierno. O como en el caso del comercio electrónico, si un
comerciante abre sus datos en la Web, los buscadores pondrán enlazar mucho mejor
sus productos, y ofrecer mejores resultados de búsqueda para su Web de comercio
electrónico. Como conclusión, podemos decir que la apertura de datos, es
beneficiosa en multitud de casos totalmente distintos, tanto como el marketing,
investigación, gobierno, etc\ldots

A raíz de todo esto, surgen multitud de vocabularios (ontologías) donde se
definen una serie de entidades y propiedades de las mismas, con las cuales
podemos relacionar nuestros datos. También existen multitud de estándares los
cuales nos permiten realizar la publicación de los datos en un formato conocido.
Por lo que, cuando vayamos a abrir nuestros datos podremos darle una estructura
conocida por todos, de forma que aquella persona que vaya a hacer uso de ellos,
pueda conocer a qué hacen referencia los mismos. De esta forma, conseguimos
darle un significado semántico a los datos que estamos mostrando en nuestra web.
Además, podremos definir relaciones entre los distintos datos, tanto los
pertenecientes a nuestros proyectos como de otras fuentes externas, lo cual se
conoce como LinkedData.

\section{Descripción del sistema actual}

En la actualidad, para el framework de desarrollo de aplicaciones web Django, no
existen aplicaciones (o éstas ofrecen pocas funcionalidades o se encuentran
obsoletas) que permitan realizar la apertura de datos de proyectos Django a
través de un menú de configuración intuitivo, haciendo uso de cualquiera de las
diferentes ontologías disponibles en internet para el marcado de los datos, como
pueden ser por ejemplo:
\begin{itemize}
    \item \textbf{Schema \cite{schema}:} proporciona un vocabulario para
        marcar los contenidos web, el cual es reconocido por los principales
        buscadores, como son: Google, Bing, Yahoo o Yandex.
    \item \textbf{FOAF \cite{foaf}:} acrónimo de Friend Of A Friend, define
        una ontología para la publicación de relaciones entre personas y datos
        en internet.
    \item \textbf{Good relations \cite{GoodRelations}:} similar a los
        anteriores, define un vocabulario específico para el comercio
        electrónico. Dicho vocabulario, ya ha sido incluido en la especificación
        de Schema.
\end{itemize}

Por otro lado, actualmente si que existen herramientas de este tipo realizadas
para otros frameworks de desarrollo de aplicaciones Web, como puede ser el caso
de una gema realizada para el framework Ruby on Rails, bajo el nombre de
EasyData\footnote{El proyecto se encuentra alojado en el repositorio
\url{https://github.com/jnillo/Linked_data}}, desarrollada también en la
Universidad de Cádiz, que ha obtenido buenos resultados. Al igual que
también conocidas aplicaciones de comercio electrónico realizadas en PHP, como
puede ser Magento o Prestashop, también están siguiendo esta tendencia y se
están adaptando al espacio de la Web semántica, vista las ventajas que ofrece.
Por lo tanto, a raíz de la utilidad del mismo y que prueba de ello cada vez más
tecnologías se están adaptando, sería interesante disponer de una herramienta de
este tipo, en un framework como Django, que cada vez está más presente en el
total de aplicaciones existentes en la Web.

\section{Objetivos y alcance del proyecto}
\label{sec:objetivos}

El objetivo principal del presente proyecto es el de proporcionar al
desarrollador de aplicaciones Django, una herramienta (conocida a partir de
ahora como EasyData/Django) la cual le permita realizar la apertura
\textbf{controlada} de los datos de su proyecto Django. Esto es el producto
final que se obtendrá al finalizar dicho proyecto, pero este proyecto se puede
desgranar en objetivos más específicos o subobjetivos, donde la unión de estos
formen la totalidad del proyecto. Estos objetivos más específicos, se pueden
dividir en tres, los cuales son:


%%%TABLA - Objetivo 001
\begin{center}
\begin{longtable}{||p{3.4cm}|p{12cm}||}
%primera parte de la tabla
 \hline \hline \bf OBJ-001 &  \bf Resolver modelos de la aplicación \\
\hline
\endfirsthead
%primera parte de la tabla por pagina
\hline \multicolumn{2}{|r|}{{Continuación de la tabla}} \\ \hline
 \hline \bf OBJ-001 &  \bf Resolver modelos de la aplicación \\
\hline
\endhead
% ultima parte de la tabla por pagina
\hline \multicolumn{2}{|l|}{{Continúa en la siguiente página}} \\ \hline
\endfoot
% ultima parte de la tabla
\endlastfoot
% DATOS
 \hline \bf Autor & José Manuel Llerena Carmona \\
 \hline \bf Descripción & Un primer objetivo será realizar una parte software,
             que sea capaz mediante introspección, de obtener los distintos
             modelos, junto con los fields y relaciones y los tipos de estos dos
             últimos que componen el proyecto Django donde se encuentra
             instalado nuestro proyecto. Esta estructura se almacenará, para
             posteriormente poder realizar la configuración de cómo se va a
             llevar a cabo la publicación de los datos.\\
 \hline \bf Subobjetivos & 
             \begin{itemize}
                 \item Confeccionar una estructura de base de datos, que
                        permita almacenar la estructura de modelos que forman el
                        proyecto Django donde se encuentra nuestra aplicación.
                 \item Desarrollar una porción de software, la cual sea capaz de
                        obtener la estructura de los modelos del proyecto Django
                        donde nos encontramos, y almacenar esta estructura en la
                        base de datos confeccionada anteriormente.
             \end{itemize}\\
\hline
\hline
\caption{\label{tab:obj001} Objetivo - 001 - Resolver modelos de la aplicación} 
\end{longtable}
\end{center}


%%%TABLA - Objetivo 002
\begin{center}
\begin{longtable}{||p{3.4cm}|p{12cm}||}
%primera parte de la tabla
 \hline \hline \bf OBJ-002 &  \bf Interfaz de configuración \\
\hline
\endfirsthead
%primera parte de la tabla por pagina
\hline \multicolumn{2}{|r|}{{Continuación de la tabla}} \\ \hline
 \hline \bf OBJ-002 &  \bf Interfaz de configuración \\
\hline
\endhead
% ultima parte de la tabla por pagina
\hline \multicolumn{2}{|l|}{{Continúa en la siguiente página}} \\ \hline
\endfoot
% ultima parte de la tabla
\endlastfoot
% DATOS
 \hline \bf Autor & José Manuel Llerena Carmona \\
 \hline \bf Descripción & Un segundo objetivo consistirá en proporcionar al
             usuario una aplicación Web de configuración, donde podrá detallar
             cómo se va a realizar la apertura de datos de su proyecto. En este
             apartado, el usuario deberá de indicar con qué entidades de los
             vocabularios disponibles en la aplicación, se relacionan los
             modelos de su proyecto, y con qué propiedades de dichas entidades,
             se relacionan los atributos y relaciones de cada uno de los modelos.
             Además se podrá especificar para cada uno de los modelos, cuales
             serán visibles y cuales no (no todos los datos son susceptibles de
             ser publicados, ya que todo proyecto puede contener datos sensibles
             o que se encuentren protegidos por leyes de protección de datos).\\
 \hline \bf Subobjetivos &
             \begin{itemize}
                 \item Proporcionar un apartado dentro del proyecto, donde el
                        usuario sea capaz de cargar los diferentes vocabularios
                        u ontologías que va a utilizar para la publicación de
                        los datos.
                 \item Proporcionar un apartado donde se listen los distintos
                        modelos que componen al proyecto Django, de tal forma
                        que para cada uno de los modelos, pudiésemos especificar
                        con qué entidades de las propuestas por las distintas
                        organizaciones se corresponden.
                 \item Un apartado más dónde para cada modelo, se pudiera
                        realizar la correspondencia (mapeo) de cada uno de los
                        fields del modelo con las propiedades de la entidad
                        asociada al modelo o cualquier otra disponible en el
                        namespace.
                 \item Un último apartado, donde para cada modelo y field o
                        relación de nuestros modelos, podamos indicar si estos
                        son visibles o no.
             \end{itemize}\\
\hline
\hline
\caption{\label{tab:obj002} Objetivo - 002 - Interfaz de configuración} 
\end{longtable}
\end{center}


%%%TABLA - Objetivo 003
%TODO: si se puede, desgranar también este apartado, cuando sepa mejor las posibles formas de mostrar los datos que se van a emplear.
\begin{center}
\begin{longtable}{||p{3.4cm}|p{12cm}||}
%primera parte de la tabla
 \hline \hline \bf OBJ-003 &  \bf Publicación de datos \\
\hline
\endfirsthead
%primera parte de la tabla por pagina
\hline \multicolumn{2}{|r|}{{Continuación de la tabla}} \\ \hline
 \hline \bf OBJ-003 &  \bf Publicación de datos \\
\hline
\endhead
% ultima parte de la tabla por pagina
\hline \multicolumn{2}{|l|}{{Continúa en la siguiente página}} \\ \hline
\endfoot
% ultima parte de la tabla
\endlastfoot
% DATOS
 \hline \bf Autor & José Manuel Llerena Carmona \\
 \hline \bf Descripción & Un tercer y último objetivo sería el de proporcionar
             al usuario, una serie de herramientas, las cuales le permitan
             realizar la publicación de los datos de su proyecto. Estas le
             permitirían al usuario utilizar los distintos tipos de notaciones
             más comúnmente usados, como pueden ser RDF/XML, RDF/Turtle,
             RDF/Ntriples, Microdata o RDFa.\\
\hline
\hline
\caption{\label{tab:obj003} Objetivo - 003 - Publicación de datos} 
\end{longtable}
\end{center}


Por lo tanto, como conclusión a todo lo comentado, el alcance del proyecto será
el de proporcionar al usuario, una aplicación Django, que le permita realizar la
apertura de los datos de forma controlada de sus proyectos Django en la Web.
Para ello, la aplicación le proporcionará al usuario, la posibilidad de usar la
mayor parte de los formatos permitidos en la Web semántica para la publicación
de datos en internet. Así como también le permitirá la carga de las diferentes
ontologías existentes en la web, de tal forma que la aplicación se adaptará a
cualquier nuevo espacio de nombres que pueda surgir en un futuro.

\newpage

\section{Definiciones}

\begin{description}
    \item[Framework] \hfill \\
        en el ámbito del desarrollo software, se entiende por framework a una
        plataforma software para desarrollar aplicaciones, la cual sirve como
        base para el desarrollo.
    \item[Modelo] \hfill \\
        es uno de los componentes del patron Modelo-Vista-Controlador. El modelo
        se encarga de representar y gestionar toda la información con la que el
        sistema trabaja.
    \item[Vista] \hfill \\
        es otro de los componentes del patron Modelo-Vista-Controlador. En el
        caso de Django, la vista hace referencia al controlador, aunque exista
        otro componente con el nombre de vista. Las vistas en Django se encarga
        de responder a eventos y tratar la información realizando peticiones a
        los modelos. Las vistas pueden utilizar las plantillas, para mostrar los
        datos generados en ellas.
    \item[Plantilla] \hfill \\
        es otro de los componentes del patron Modelo-Vista-Controlador. En el
        caso de Django, la plantilla hace referencia a la vista del patrón MVC.
        La plantilla se encarga de presentar los datos en un formato adecuado
        para la interacción.
    \item[Ontología o Namespace] \hfill \\
        dentro de la aplicación que se va a desarrollar, se conoce por ontología
        o namespace, a cada uno de los vocabularios definidos que están
        disponibles para que los usuarios realicen el marcado del contenido de
        sus datos.
    \item[Python] \hfill \\
        \cite{python} es el lenguaje de programación que se va a utilizar para el desarrollo
        del proyecto. Este lenguaje de programación es multiparadigma, ya que
        permite la utilización de diferentes paradigmas de programación
        (orientado a objetos, funcional o imperativa). Este lenguaje de
        programación también se caracteriza por ser multiplataforma y de código
        abierto. Para más información se puede consultar la web oficial de
        \href{http://es.wikipedia.org/wiki/Python}{Python}.
    \item[Django] \hfill \\
        se trata de un framework de desarrollo web de código abierto, el cual se
        utilizará para la realización del proyecto. Este framework está escrito
        en el lenguaje de programación Python y está basado en el modelo MVC
        (Modelo Vista Controlador).
    \item[RDF] \hfill \\
        \cite{rdfdoc} es el acrónimo de Resource Description Framework. Está diseñado por el
        World Wide Web Consistorium como un modelo de datos para metadatos. Se
        usa para la descripción conceptual y modelado de la información para
        recursos web.
    \item[RDFa] \hfill \\
        es el acrónimo de Resource Description Framework in Attributes. Se trata
        de una serie de extensiones para lenguajes HTML, XHTML y ciertos
        documentos basados en XML recomendadas por W3C, de forma que se permita
        dotar de contenido semático a los documentos web.
    \item[Microdata] \hfill \\
        es una especificación HTML de WHATWG (Web Hypertext Application
        Technology Working Group), para enlazar metadatos al contenido existente
        en páginas web.
    \item[Web semántica] \hfill \\
        es un conjunto de actividades promovidas por el W3C con la finalidad de
        crear tecnologías que permitan la publicación de datos legibles por
        aplicaciones informáticas.
\end{description}

\section{Organización del documento}

Este documento tiene como finalidad, describir todo el proceso de desarrollo de
la aplicación Django para la apertura de datos de proyectos y su posterior
publicación, empezando por la descripción de los objetivos iniciales y
requisitos esenciales que deberá reunir dicho proyecto. Se mostrará una
planificación del desarrollo del proyecto, incluyendo costes estimados y
diagramas de Gantt que ilustren el avance del proyecto en el tiempo.

En este documento también se plasmará todo el ciclo de vida de desarrollo, desde
la fase de análisis y diseño de los requisitos del proyecto, definiendo los
distintos casos de usos, diagramas con la organización de los datos, actores
implicados o diagramas de secuencia e interacción que pudieran resultar del
mismo. Siguiendo por la implementación del mismo y las pruebas realizadas, las
cuales verifiquen la calidad y robustez del mismo. Además, en este documento
también se comentarán las posibles mejoras o ampliaciones futuras que pudieran
considerarse.

Por último, proporcionará al usuario tanto un manual de uso, como un manual de
instalación y explotación del mismo, de tal forma que cualquier usuario sin
conocimientos previos de la aplicación, pueda hacer uso de ella y sacarle el
máximo partido posible.
