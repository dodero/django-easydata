% ------------------------------------------------------------------------------
% Este fichero es parte de la plantilla LaTeX para la realización de Proyectos
% Final de Grado, protegido bajo los términos de la licencia GFDL.
% Para más información, la licencia completa viene incluida en el
% fichero fdl-1.3.tex

% Copyright (C) 2012 SPI-FM. Universidad de Cádiz
% ------------------------------------------------------------------------------

\thispagestyle{empty}

\noindent \textbf{\begin{Large}Resumen\end{Large}} 
\newline
\newline
\noindent Hoy en día es mucha la información que circula a través de internet, y
aunque la gran mayoría sea comprensible por el ser humano, las máquinas no
disponen de la capacidad de razonamiento necesario para comprender en muchos de
los casos de qué trata dicha información o con qué está relacionada. De esta
forma, mediante el marcado de nuestros datos haciendo uso de vocabularios
específicos, podemos dotar a nuestro contenido web de un mayor valor semántico
(web semántica), de tal forma que estemos indicando explícitamente qué es la
información que estamos publicando.

\noindent En el presente proyecto titulado Apertura de datos en aplicaciones
Django, se ha desarrollado una herramienta para el framework de desarrollo web
Django, el cual se encuentra desarrollado bajo el lenguaje de programación
Python. Esta herramienta o plugin, se encargará de facilitar al usuario la
publicación controlada de los datos de sus proyectos Django en internet,
haciendo uso de los vocabularios u ontologías disponibles en internet para tal
finalidad.

\noindent En la web están disponibles multitud de ontologías disponibles para
los usuarios que deseen hacer uso de ellas. Cada una de ellas relacionadas con
una temática diferente o con muchas temáticas a la vez, como por ejemplo puede
ser, la música, la ingeniería, el cine, electrónica, etc\ldots

\noindent A día de hoy, multitud de organizaciones, tanto públicas como privadas,
usuarios y software están adaptando sus productos a la web semántica para la
apertura de sus datos. Un ejemplo bastante claro es el Open Government (o
gobierno abierto), donde se defiende que los temas del Gobierno y
Administración Pública se abran a los ciudadanos, de tal forma que se transmita
transparencia en sus gestiones. También en el ámbito del comercio, aplicaciones
bastante populares de comercio electrónico, están adaptando o han adaptado sus
aplicaciones para la publicación de sus datos, confiriendo de esta forma una
mayor proyección de los productos.

\noindent A raíz de todo esto comentado, surge la idea de realizar una
herramienta para un framework de desarrollo web cada vez más usado, que permita
adaptar los proyectos existentes a la publicación de datos, sin necesidad de
redesarrollar los proyectos por completo.

\noindent Por lo tanto, la herramienta que se ha desarrollado en el presente
proyecto, permitirá a partir de sus datos realizar un etiquetado de los mismos
para luego publicar estos en internet. Para desarrollar esto, la presente
herramienta ofrece al usuario las siguientes funcionalidades:
\begin{itemize}
    \item Captación de los datos de los que está compuesto nuestro proyecto
        Django.
    \item Carga de espacios de nombres y ontologías definidas en la web que
        sirvan para el propósito de los datos que se desean publicar.
    \item Posibilidad de elegir qué datos van a ser publicados y cuales no lo
        serán.
    \item Componente para la interrelación de los espacios de nombres y los datos de nuestro proyecto.
    \item Componente para la publicación de nuestros datos en diferentes
        formatos existentes.
\end{itemize}

\noindent {\bf Palabras clave:} OpenData, LinkedData, RDF, RDFa, Microdata,
ontología, namespace, \mbox{RDF/Ntriples}, RDF/Turtle, XML, Django, Python, SPARQL,
D2Rq.
